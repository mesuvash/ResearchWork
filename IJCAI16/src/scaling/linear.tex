% Linear methods, LRec and SLIM, work very well in practice. However, they require solving multiple linear regression for a very large design matrix $R$. 

%Although the algorithm is distributable, the number of users and items can be in the order of millions in a real world problems making it computationally expensive. 
% Despite of superior performance, LRec's memory requirement is quadratic and thus limiting its applicability in real-world large scale recommendation.
% Despite LRec's superior performance, the memory requirement is quadratic which limits its applicability in a large scale datasets.

% \subsubsection{Via SVD}
Despite of its superior performance, the applicability of Linear methods on real world large scale dataset are constrained by their computational cost. Linear methods involve solving  a large number of regression subproblems on a huge design matrix $\R$ making it extremely challenging on real world applications where the number of users and items is in millions.

In this section, we propose an algorithm to scale linear methods on large scale datasets. First, we define $L_2$ regularized linear model 
\begin{equation}
\label{eqn:squaredLinear}
\underset{\W}{\mathrm{argmin}}|| \R - \R \W ||_F^2 + \frac{\lambda}{2} || \W ||_F^2 
% \inf_{ \W  } || \R - \R \W ||_F^2 + \frac{\lambda}{2} || \W ||_F^2 
\end{equation}

We observe that (\ref{eqn:squaredLinear}) is a  standard multiple linear regression problem with analytical solution as
% We can solve Equation~\ref{eqn:squaredLinear} analytically as
\begin{equation}
\W = (\R^T \R + \lambda \I)^{-1} \R^T \R
\end{equation}
However, the  memory requirement, computational cost and  numerical instability in computing the inverse, $(\R^T \R + \lambda \I)^{-1}$ , makes the analytical solution infeasible. Similarly, if we apply gradient based optimization, it becomes computationally expensive as it involves solving $n$ linear regression. \todob{This is a scaling up paper. Be precise about the computational cost.}
% Gradient based optimization doesn't require computation of the inverse. However, it involves solving $n$ linear regression problem which is computationally expensive.

To address the computational and numerical issues, we reformulate the linear model with additional rank constraints \todob{Use s.t. instead of where} \todob{Use $\ll$ instead of $< <$.}
\begin{align}
\label{eqn:reg-rank}
\begin{split}
\underset{rank(\W) \le k}{\mathrm{argmin}}  \left \| \R - \R\W\right \|_F^2 + \lambda \left \|  \W \right \|_F^2 \\
where,\ k\ <<\ n 
\end{split}
\end{align}

% $ I \in \mathbb{R}^{n \times n}$ would be the trivial solution for W, however, the rank constraint prevents W from the trivial solution as $\{rank(I) = n\} > p $.

For $\lambda = 0$ , the optimal solution is given by SVD as
% Even though we can't directly solve the eqn~\ref{eqn:reg-rank}, the optimal solution when $\lambda = 0$ is given by SVD as 
   \begin{align}
   \label{eqn:SVDOptimal}
   \W = \Q_k \Q_k^ T
   \end{align}
where  $\Q_k$ is given by truncated SVD 
   \begin{align*}
   % R =\ P \Sig Q^T \\
   \R \approx \P_k \Sig_k \Q_k^T
   \end{align*}
However, the key limitation of the SVD solution is that it corresponds to an unregularized model, hence highly susceptible to overfitting. To incorporate model complexity control via regularization, we constrain (\ref{eqn:squaredLinear}) further by factoring $W\in span(Q_k)$ by writing $W = Q_k Y$ which gives the following optimization objective

\begin{align}
\label{eqn:i-reg-rank-low}
\underset{\Y} {\mathrm{argmin}}  \left \| \R - \R \Q_k \Y \right \|_F^2 + \lambda \left \|  \Q_k \Y \right \|_F^2 
\end{align}

Since $\Q_k$ is orthonormal, we get $\left \| \Q_k \Y \right \|_F = \left \|\Y\right \|_F$. Hence~\ref{eqn:i-reg-rank-low} is equivalent to 
\begin{align}
% \begin{split}
\underset{\Y}{\mathrm{argmin}} \left \| \R - \R\Q_k\Y\right \|_F^2 + \lambda \left \|  \Y \right \|_F^2 
% \end{split}
\end{align}

For a resonable rank $k$, we can efficiently solve (\ref{eqn:i-reg-rank-low}) with the analytical solution 
\begin{align*}
\Y = (\Q_k^T \R^T\R\Q_k +\lambda I)^{-1} \Q_k^T \R^T \R
\end{align*}
\todob{This is a scaling up paper. Be precise about the computational cost.}

In other words, equation (~\ref{eqn:i-reg-rank-low}) corresponds to the linear Autoencoder ~\cite{Sedhain:2015} where the input-hidden layer weight is initialized with the rank-k orthogonal basis of the original matrix. 

We refer to (\ref{eqn:i-reg-rank-low}) as I-\LinearLow as it corresponds to item-item model. Similarly, we can define a user-user model, U-\LinearLow, in (\ref{eqn:u-reg-rank-low})
\begin{align}
\label{eqn:u-reg-rank-low}
% \begin{split}
% \underset{rank(Y) = k}{\mathrm{argmin}}  \left \| R^T - R^TP_kY\right \|_F^2 + \lambda \left \|  P_kY \right \|_F^2 \\
\underset{\Y}{\mathrm{argmin}}  \left \| \R - \R\P_k\Y\right \|_F^2 + \lambda \left \|  \Y \right \|_F^2 
% \because  \left \|  P_kY \right \|_F^2 = \left \|  Y \right \|_F^2
% \end{split}
\end{align}

Although low rank multiple-regression can be solved efficiently, it requires computationally expensive dimensionality reduction. However, the recent advances in randomized algorithms for approximate dimensionality reduction~\citep{halko2011} have made dimensionality reduction very efficient on large scale matrices. In this work, we use randomized SVD~\citep{halko2011} for low dimensional projection as summarized in Algorithm~\ref{algo:RSVD}.

As discussed earlier, recommending similar items is very prevalent in real-world recommender systems.
In I-\LinearLow model, item-item similarity can be recovered as $Q_kY$. 

\begin{equation*}
\end{equation*}


