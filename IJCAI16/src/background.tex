
 % summarizes the commonly used symbols.
%Let $\userSet = \{ 1, \ldots, \numUsers \}$ denote a set of users, and $\itemSet = \{ 1, \ldots, \numItems \}$ a set of items.
Let $\userSet$ denote a set of users, and $\itemSet$ a set of items,
with $\numUsers = | \userSet |$ and $\numItems = | \itemSet |$.
In OC-CF, we have a purchase\footnote{We use the word ``purchase'' simply for the purposes of exposition.} 
matrix $\R \in \{ 0, 1 \}^{\numUsers \times \numItems}$. %, where $\numItems$ is the number of items and $\numUsers$ the number of users.
We use $\R_{\ui}$ to refer to the purchase status for the user $\u$ and item $\i$.
We use $\R_{:\i}$ to denote the indicator vector of each users' purchase of an item, and similarly $\R_{\u:}$ to denote the vector of a user's preference for each item $i$.
We denote by  $\ratedSet{\u}$ the set of the items purchased by user $\u$. The goal in OC-CF is to learn a recommender, which for our purposes is simply a matrix $\RHat \in \Real^{\numUsers \times \numItems}$. %of the same size as $\R$.
We call $\RHat$ the \emph{recommendation matrix}. The notations are summarized in table~\ref{tbl:glossary}. 
%As the entries are real-valued, for each user $u$, we can sort the entries of $\RHat_{\u:}$ to obtain a predicted ranking over items.
%We will use $\predUserItem$ to denote the predicted score for a given (user, item) pair.
% We evaluate $\RHat$ assuming we are in the \emph{top-$N$ recommendation} scenario \citep{Deshpande:2004}.
%Here, the interest is in finding $\RHat$ such that the head of the ranked list for each user comprises items she will enjoy. %, as measured by metrics such as Precision@K.
% This is more realistic than the \emph{rating prediction} scenario, where the interest is in accurately predicting the (typically ordinal) entries of $\R$ \citep{Marlin:2004}. %, as measured by \eg RMSE.
%One can use techniques from the latter for top-$N$ recommendation, but they may underperform \citep{Cremonesi:2010}.

\begin{table}[t]
	\centering
	\caption{Commonly used symbols.}
	\label{tbl:glossary}
	\resizebox{0.5\textwidth}{!}{%

	\begin{tabular}{@{}llll@{}}
	\toprule
	\toprule
	\textbf{Symbol} & \textbf{Meaning} & \textbf{Symbol} & \textbf{Meaning} \\
	\toprule
	$\userSet$ & Set of users & $\R$ & Purchase matrix \\
	$\itemSet$ & Set of items & $\RHat$ & Recommendation matrix \\
	$m$ & Number of users & $\ratedSet{\u}$	& Items purchased by $\u$ \\
	$n$ & Number of items & $\Sim$ & Item similarity matrix \\
	\bottomrule
	\end{tabular}
	}
\end{table}

% The challenge in OC-CF is the lack of examples of explicit negative preference.
% %We may assume that if a user purchased an item, \ie $\R_{\ui} = 1$, this indicates a positive preference.
% %However, if a user has not purchased an item, \ie $\R_{\ui} = 0$, this does not necessarily indicate a negative preference, as the user may simply be unaware of the item.
% While we may assume that a user purchasing an item is an indication of positive preference, a user not purchasing an item does not necessarily indicate a negative preference: the user may simply be unaware of the item.
% Henceforth, we say that $(\u, \i)$ pairs with $\R_{\ui} = 1$ denote ``known positive preference'', while pairs with $\R_{\ui} = 0$ denote ``unknown preference''.
In the rest of this section we review some existing approaches to OC-CF.
%
\subsection{Neighbourhood methods}
\label{sec:knn}

In (item-based) neighbourhood methods, we produce a recommendation matrix of the form
\begin{equation}
\label{eqn:knn}
%\predUserItem = \sum_{\i' \in \itemSet} \R_{\u \i'} \cdot \Sim_{\i' \i},
\RHat = \R \Sim
\end{equation}
where $\Sim \in \Real^{\numItems \times \numItems}$ is some \emph{item-similarity matrix}.
% That is, for a user $\u$, one scores an item $\i$ as the \emph{sum of $\i$'s similarity to all other items that $\u$ enjoys}.
%Given an item $i$, the nonzero entries of $\Sim_{i:}$ are known as the \emph{neighbours} of the item.
Typically, one uses a predefined matrix $\Sim$ that relies on $\R$. A popular example is cosine similarity \citep{Sarwar:2001, Linden:2003}, 
$$ \Sim_{\i' \i} = \frac{ \R_{:\i}^T \R_{:\i'} }{|| \R_{:\i} ||_2 || \R_{:\i'} ||_2}. $$
% Other examples are the Pearson correlation, Jaccard coefficient and conditional probability \citep{Deshpande:2004}. 
It is typical to sparsify $\Sim$ so that its columns only keeps top-k similar items. 
%the largest entries, \ie to compute the $k$-neighbourhood graph based on $\Sim$.
Neighbourhood methods are attractive for several reasons.
They are simple to implement, efficient 
%(as the computation of (\ref{eqn:knn}) requires $O( | \ratedSet{\u} | )$ time),
and interpretable.
%Further, they may also be applied as-is to OC-CF problems \citep{Linden:2003, Deshpande:2004}, unlike most other popular CF models such as matrix Factorization (to be discussed subsequently).
However, they are unable to adapt to the characteristics of the data as they rely on a fixed $\Sim$ that is not learned from the new data \citep{Koren:2008b}. Furthermore, recommendation performance can be quite sensitive to the choice of $\Sim$.
% However, their biggest drawback is that they rely on a fixed $\Sim$ that is not learned from data \citep{Koren:2008b}.
% Recommendation performance can be quite sensitive to the choice of $\Sim$, and the choice generally depends on the problem domain.
% Therefore, neighbourhood methods are unable to adapt to the characteristics of the data at hand.

\subsection{Matrix Factorization}
% \subsubsection{Matrix Factorization}

Matrix Factorization methods are the \emph{de facto} approach to collaborative filtering with explicit feedback. The basic idea is to embed users and items into some shared latent space, with the aim of inferring complex preference profiles for both users and items. 
Weighted matrix factorization (WRMF) for one class collaborative filtering~\cite{Hu:2008, Pan:2009} optimizes
\begin{equation}
\label{eqn:wrmf}
\min_{ \A,\B  } \sum_{ u \in \userSet, i \in \itemSet } \J_{\ui} ( \R_{\ui} - \A_u^T \B_i )^2 + \frac{\lambda}{2}  ( ||\A||_F^2 + ||\B||_F^2 )
\end{equation}
where 
$\J \in \Real_+^{\numUsers \times \numItems}$ is some pre-defined weighting matrix  and 
\begin{equation}
\label{eqn:wrmf-weight}
\J_{\ui} =  \indicator{ \R_{\ui} = 0 } + \alpha \indicator{ \R_{\ui} > 0 } 
\end{equation}
where $\alpha$ assigns an importance weight to the observed interaction.
WRMF uses Alternating Least Squares(ALS) method for optimization. The solution for $A$ and $B$ in each step of ALS is given by (~\ref{eqn:ALS_A})
%For instance, equation(~\ref{eqn:ALS_A}) gives ALS solution of $A$ at the given iteration.
% shown in~\ref{ and vice-versa for $B$
\begin{align}
	\label{eqn:ALS_A}
	\begin{split}
	\A_u = (\B^T \J^u \B + \lambda I)^{-1} \B^T \J^u \R_{u:} \\
	\B_i = (\A^T \J^i \A + \lambda I)^{-1} \A^T \J^i \R_{i:} 
	\end{split}
\end{align}
where, $\J^u \in \mathbb{R}^{n \times n}$ and $\J^i \in \mathbb{R}^{m \times m}$ are diagonal matrix such that $ \J^u_{ij} = \J_{ui}$ and  $\J^i_{uu} = \J_{ui}$. In each iteration of ALS, due to the weighting, we need to compute inverse for each user and item as shown in (\ref{eqn:ALS_A}). This makes WRMF computationally expensive compared to unweighted matrix factorization.


\subsection{Linear Recommenders}

Linear methods~\cite{Ning:2011, Sedhain:2016} learn the similarity metric from the data. SLIM~\citep{Ning:2011} views the recommendation as learning item-item similarity and learns an item-similarity matrix $\W \in \Real^{\numItems \times \numItems}$ via
\begin{equation}
\label{eqn:slim}
\min_{ \W \in \C } || \R - \R \W ||_F^2 + \frac{\lambda}{2} || \W ||_F^2 + \mu || \W ||_1,
\end{equation}
where $\lambda, \mu > 0$ are appropriate constants, and
\begin{equation}
\label{eqn:slim-constraint}
\C = \{ \W \in \Real^{\numItems \times \numItems} \colon \text{diag}( \W ) = 0, \W \geq 0 \}.
\end{equation}
Here, $|| \cdot ||_1$ denotes the elementwise $\ell_1$ norm of $\W$ so as to encourage sparsity, and the constraint $\text{diag}( \W ) = 0$ prevents a trivial solution of $\W = \id_{\numItems \times \numItems}$. SLIM is equivalent to an item-based neighbourhood approach where the similarity matrix $\Sim = \W$ is \emph{learned} from the data.
%
In a related linear model for OC-CF~\citep{Sedhain:2016}, the authors report very good performance, but the proposed method is computationally expensive, which restricts its applicability in real world problems.
% LRec solves
% \begin{equation}
% \label{eqn:lrec}
% \min_{ \theta } \sum_{\u \in \userSet} \sum_{\i \in \itemSet} \ell( \y^{(\u)}_{\i}, \X_{\i :} \w^{(\u)} ) + \mathrm{\Omega}(\theta),
% \end{equation}
% %where $\ell(y, v) = \log(1 + e^{-yv})$ is the logistic loss,
% where $X = R^T$; $\ell$ is some convex loss function such as squared or logistic loss; 
% $\theta = \{ \w^{(\u)}  \}_{\u \in \userSet}$. 

% For squared loss, we minimize the following objective 
% \begin{equation}
% \label{eqn:lrecsquared}
% \min_{ \W  } || \X - \X \W ||_F^2 + \frac{\lambda}{2} || \W ||_F^2 
% \end{equation}

Linear methods are attractive for several reasons. They have superior performance, and unlike neighborhood methods, they adapt with the data as the parameters are learned from data itself. Furthermore, the recommendations are easily interpretable. However, linear methods can be computationally expensive as they require solving a large number of regression subproblems with a big design matrix $\R$. This can scale quadratically with the order of $\R$, or worse.

% The nonnegativity constraint encourages interpretability, but \citet{Levy:2013} demonstrated that good performance can be achieved without it.

% Given a learned $\W$, SLIM produces a recommendation matrix
% $$\RHat(\theta) = \R \W.$$
% Thus, SLIM is equivalent to an item-based neighbourhood approach where the similarity matrix $\Sim = \W$ is \emph{learned} from data.

% \subsubsection{LRec}
% LRec \citep{LRec}  decomposes the OO-CF into learning a linear model per user for personalized recommendation. 
% Let's define a matrix $\X \in \Real^{\numItems \times \numUsers}$ by
% $$\X = \R^T,$$
% and, for each $\u  \in \userSet$, define a vector $\y^{(\u)} \in \PMOne^{\numItems}$ that is the same as $\R_{\u:}$ except with zeros in $\R_{\u:}$ replaced by -1:
% \begin{align*}
% \y^{(\u)} &= 2 \R_{\u:} - 1.
% \end{align*}

% Then, LRec solves
% \begin{equation}
% \label{eqn:lrec}
% \min_{ \theta } \sum_{\u \in \userSet} \sum_{\i \in \itemSet} \ell( \y^{(\u)}_{\i}, \X_{\i :} \w^{(\u)} ) + \mathrm{\Omega}(\theta),
% \end{equation}
% %where $\ell(y, v) = \log(1 + e^{-yv})$ is the logistic loss,
% where $\ell$ is some convex loss function such as squared loss, logistic loss; 
% $\theta = \{ \w^{(\u)}  \}_{\u \in \userSet}$. 

% For squared loss, we minimize the following objective 
% \begin{equation}
% \label{eqn:lrecsquared}
% \min_{ \W  } || \X - \X \W ||_F^2 + \frac{\lambda}{2} || \W ||_F^2 
% \end{equation}
%Each $\w^{(\u)} \in \Real^{\numUsers}$, and so we can equivalently think of $\theta = \{ \W \}$ for some $\W \in \Real^{\numUsers \times \numUsers}$, with $\w^{(\u)} = \W_{:\u}$.
% LRec employs squared loss and logistic loss
% Linear loss and logistic loss are favored due to the availability of the efficient solvers. LRec allows fine tuing per user level and efficient hyperparameter selection by cross-validating on the subset of the users.

%

\subsection{Randomized SVD}
SVD is the archetypal matrix factorization algorithm and has been widely used in machine learning for dimensionality reduction. However, SVD is computationally expensive and not scalable to large scale datasets. It has been recently shown that SVD can be significantly scaled up, at a negligible cost in performance, by randomization~\citep{halko2011}. We will describe the randomized SVD algorithm in the next section, in the context of the OC-CF problem.

Randomized SVD has been applied to matrix factorization \citep{Tang:2013} 
(but not in the OC-CF setting that we are considering). 
The authors compute the rank-k randomized SVD of the matrix $\R$,
\begin{align*}
	\R \approx \P_k \Sig_k \Q^T_k
\end{align*}
where, $\P_k \in \mathbb{R}^{m \times k}$, $\Q_k \in \mathbb{R}^{n \times k}$ and $\Sig_k \in \mathbb{R}^{k \times k}$. Given the truncated SVD solution, they initialize the item latent factor with the SVD solution and solve 
\begin{align}
\label{eqn:I-MF-RSVD}
\underset{\A}{\mathrm{argmin}} & \left \| \R - \A^T\B\right \|_F^2 + \lambda \left \|  \A \right \|_F^2   \\
s.t.\ & \B = \Sig_k^{\frac{1}{2}} \Q^T_k \nonumber
\end{align}
Similarly, if the matrix $\A$ is fixed instead of the matrix $\B$, the objective becomes
\begin{align}
\label{eqn:U-MF-RSVD}
% \label{eqn:U-MF-RSVD}
\underset{\B}{\mathrm{argmin}} & \left \| \R - \A^T\B\right \|_F^2 + \lambda \left \|  \B \right \|_F^2 \\
s.t.\ & \A = \P_k \Sig_k^{\frac{1}{2}} \nonumber
\end{align}
We refer to (\ref{eqn:U-MF-RSVD}) and (\ref{eqn:I-MF-RSVD}) as U-MF-RSVD and I-MF-RSVD respectively. As we will see in the results, the performance of these two models can vary significantly.
