In this section, we qualitatively analyze the item-item similarities learned by I-\LinearLow model. For this purpose, we use \Fotolia dataset\footnote{\scriptsize The dataset sharing agreement with the provider restricts us from reporting the statistics and quantitative results. Hence, we do not report summary statistics and quantitative results on this dataset}. This dataset consists of user interactions i.e. clicks on various image categories.
% users clicks on various image categories. 
We choose this dataset because unlike other datsets, items in \Fotolia dataset are easy to comprehend.
% Since the similarity metric is an important aspect of personalization, we evaluate the item-item similarities learned by I-\LinearLow  model on the \Fotolia dataset. 

In Table~\ref{tbl:similarity-evaluation}, we show top-3 similar items for a small subset of items learned by I-\LinearLow model. We observe that  I-\LinearLow learns meaningful and explainable similarities making it very applicable in recommending similar items.

